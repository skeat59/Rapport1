
\documentclass [11pt]{report}

\usepackage{fancyhdr}
\usepackage [french]{babel}

\usepackage[utf8]{inputenc}
\usepackage[T1]{fontenc}
\usepackage{textcomp}
\usepackage{graphicx}
\usepackage{titlepic}
\usepackage{boxedminipage}
\usepackage{listings}
\usepackage{minitoc}
\usepackage{footmisc}
\usepackage{color}
\usepackage{graphicx}

\usepackage{eso-pic}

\makeatletter
\newlength\@tempdim@x
\newlength\@tempdim@y
% structure des commandes :
%   #1 = deplacement selon x
%   #2 = deplacement selon y
%   #3 = texte à mettre
\newcommand\AtUpperLeftCorner[3]{%
\begingroup
\@tempdim@x=0cm
\@tempdim@y=\paperheight
\advance\@tempdim@x#1
\advance\@tempdim@y-#2
\put(\LenToUnit{\@tempdim@x},\LenToUnit{\@tempdim@y}){#3}%
\endgroup
}
\newcommand\AtUpperRightCorner[3]{%
\begingroup
\@tempdim@x=\paperwidth
\@tempdim@y=\paperheight
\advance\@tempdim@x-#1
\advance\@tempdim@y-#2
\put(\LenToUnit{\@tempdim@x},\LenToUnit{\@tempdim@y}){#3}%
\endgroup
}
\newcommand\AtLowerLeftCorner[3]{%
\begingroup
\@tempdim@x=0cm
\@tempdim@y=0cm
\advance\@tempdim@x#1
\advance\@tempdim@y#2
\put(\LenToUnit{\@tempdim@x},\LenToUnit{\@tempdim@y}){#3}%
\endgroup
}
\newcommand\AtLowerRightCorner[3]{%
\begingroup
\@tempdim@x=\paperwidth
\@tempdim@y=0cm
\advance\@tempdim@x-#1
\advance\@tempdim@y#2
\put(\LenToUnit{\@tempdim@x},\LenToUnit{\@tempdim@y}){#3}%
\endgroup
}
% ajout de texte ou d'images en haut à gauche, en haut à droite, etc.
\AddToShipoutPicture{%
\AtLowerRightCorner{3cm}{1cm}{\includegraphics[scale=0.20]{images/LogoGroupe.png}}% image en bas à droite
}
\makeatother

\pagestyle{fancy}



\title{
	\includegraphics[scale=0.43]{images/Logojeu.png}
	 \\\vspace{20mm}
	\textbf{\Huge \itshape Cahier des charges }
	}




\author{ \\\vspace{2mm}
	Thibault Gdalia\\\vspace{2mm}
	Florent Youinou\\\vspace{2mm}
	Mathilde Laplaze\\\vspace{2mm}
	Vincent Baille \\\vspace{30mm}
	}


\date{17 janvier 2014}


\begin{document}

\renewcommand{\baselinestretch}{0.001}
\maketitle
\tableofcontents

\newpage


\textbf{{\Huge Introduction}}\\
\\
\\
\indent Nous sommes la Team Girafe, composé de Mathilde "Mattou" Laplaze, Vincent "Vincae" Baille, Florent "T4ze" Youinou, Thibault "Skeat" Gdalia. Nous produisons un runner 2D, ou le joueur est dans la peau d'un oiseau très gourmand, qui a besoin de manger beaucoup de sucre pour avoir la force de voler. Le jeu est composé de différents mode de jeu, le mode solo où vous devez parcourir les différentes maps sans sortir de l'écran (car si cela se produit, c'est que vous mort, it's too bad), et d'un mode multijoueur ou vous êtes sur une map infini, le but étant d'aller le plus loin possible (le nombre de mètres étant comptabilisé) une fois que vous êtes mort (ça arrive a tous un jour malheureusement) le nombreux de mètres est envoyé vers notre base de donner est le classement des joueurs est mis a jour sur notre site internet.

\chapter{Avancement du Projet}
	\section{Moteur Physique}
		Le moteur physique est vraiment la base de notre jeu, donc tout naturellement il était notre plus grand centre d'attention durant cette première partie de notre projet. nosu avons fait en sorte qu'il soit efficace et que l'on puisse le moduler aisément modulable pour que l'on puisse s'amuser a changé les propriétés physique de chaque map.\\
		\indent Au début nous avions fait une collision par rectangle, mais notre oiseaux est rond... Nous sommes d'accord que ce n'est pas une bonne idée. Par la suite nous avons remédier a ce problème qui n'en est pas réellement un. Nous nous sommes orienté vers une collision par pixel qui s'avère être beaucoup plus intéressante pour nous.
	\section{Réseau}

	\section{Son}
	Nous savons que tout bon jeu est accompagné d'une vraie bibliothèque de sons, c'est pourquoi Mattou s'est vraiment concentré sur cette aspect du projet il a fallu tout d'abord qu'elle recherche les musique et les effets sonore adéquat pour que l'ambiance lorsque vous êtes dans le jeu soit optimale. Puis nous voulions que ce soit trés structuré dans le projet pour ne pas avoir des sons en vrac dans nos Contents, notre solution: utiliser Xact. 
	
	\newpage
	\section{\'Editeur de Map}
	Nos Maps sont basé sur des fichiers textes (en .txt) que nous renommons en .lvl pour level (sans blagues). Le fichier est composé de: sur la première ligne le nombre d'élement pouvant se trouvé sur une même colonne, sur la deuxième ligne le nombre d'élement sur une ligne, puis sur le reste c'est la définition de la map % photo du .lvl % 
	les "0" correspondent à un espace vide, les 4 sont eux des obstacles et 3 sont les fameux bonbons permettant de regagner de l'énergie.
	\indent Il est bien entendu que nous n'allons pas écrire toutes nos maps à la main cela serait trop long et très très lassant. Donc nous avons créé un éditeur de maps qui écrira dans le fichier .lvl pour nous avons juste a cliquer un peu partout pour avoir une map (SUPER) %photo éditeur 
	 
	\section{Site Web}
	Notre site a été entièrement réaliser à la main, vous pouvez retrouver une présentation rapide de notre projet accompagné d'images illustrant les différentes parties de notre jeu. Chaque membres du groupes a sa page, avec une petite présentation et une photo (la classe). Passons au choses sérieusement vous pouvez télécharger nos différents rapports, en LaTeX ou en PDF, ainsi que la version du code source présenter lors des différentes soutenances. Vous pouvez vous inscrire, car il est obligatoire d'avoir un compte chez nous pour accéder au multijoueurs (avec numéro de carte bleu tout ça tout ça... histoire de rembourser tout nos frais de l'année), et le classements de nos joueurs. Une page de contact est disponible afin de vous permettre de nous faire part de vos commentaires (pas de vos problèmes, nous codons bien donc il n'y a pas de buggs).
	\section{Graphique}
	La partie a était compliqué suite au départ d'Adrien qui n'était pas prévue, il était le plus expérimenté d'entre nous avec les logiciels de graphismes (Photoshop en l'occurrence). Il fallu très vite apprendre les bases de ce logiciel afin de pouvoir réutiliser les travaux déjà produit par Adrien. Nous avons donc du faire face a ce petit problème car un jeu avec des graphismes mal fini n'ai pas très attirant et le savons bien.\\
	\indent Nous n'avons pas de personne assigner au graphismes, chacun crée les graphismes ont il a besoin au fur et à mesure que nous avançons dans nos parties. Ce n'est pas forcément très efficace mais cela évite qu'un des membres se consacre entièrement au graphismes au détriment du code ce qui n'est pas le but de ce projet. Nous souhaitons que chaque membres puisse apprendre ce qu'il souhaite, c'est pour cela que nous sommes parfois peut-être trop nombreux sur une tache, mais cette méthode a l'avantage d'impliquer l'ensemble de l'équipe a 100\% dans le projet.\\*
	\indent Nous avons tout de même défini une chatre dans nos graphismes pour ne pas tomber dans un trop gros décalage entre deux partie du jeu. Nous avons donc des modèles de départ pour partir sur les mêmes bases, puis chacun les modifie, tout en montrant aux membres de l'équipe pour qu'ils puissent donner leur avis,et voir si cela reste en accord avec le reste du projet
\chapter{Et Après?}
	\section{Map Editor 2.0}
	L'éditeur de map serat intégrer au jeu afin que vous puissiez créer vos propres map et ainsi étendre la durée de vie de notre jeu (qui est déjà infini), vous aurez aussi mettre au défi vos amis en partageant vos créations, ce qui rendra le jeu quelque peu plus addictif.
	\section{Réseaux Multijoueur}

	\section{Nouveaux Personnage}
	Nous savons que nos bel oiseau est très attachant, nous avons décider de rajouter de nouveau personnages, qui auront des caractéristiques physiques différentes, pour que vous puissiez choisir choisir un personnage qui correspondent le mieux a votre style de jeu.
	\section{Graphismes}
\newpage
\textbf{{\huge Conclusion}}
\end {document}